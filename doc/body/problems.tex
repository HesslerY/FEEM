\chapter{开发问题记录}

\section{开发技巧}   

\subsection{Qt包含目录}
在使用Qt的某一个模块的时候,需要include相应的头文件,为了告诉编译器头文件的具体位置,需要在pro文件中指定需要使用的相关模块,这样qmake会对pro文件进行处理,生成相应的路径。也可以指定不是用某些库。

常见的模块有:core、gui、widgets等,如果想要包含private的头文件,那么就需要接着添加core-private、gui-private、widgets-private等。具体的效果可以使用VS打开pro文件,然后查看相应的头文件设置信息。          

\subsection{预编译文件}
当一个工程项目很大时,会出现很多的头文件,这个时候,把他们全部一次性加入到工程当中是一件非常繁琐和头疼的事情。在Qt当中,可以进行这样的方便设置。将每一个模块的代码(包括h文件和cpp文件)单独放置在一个文件下面,然后再新建一个pri文件,在pri文件当中设置需要include的文件列表。这样,当我们需要使用这个模块时,不需要再把源代码加入到工程当中,只需要把该pri文件include到pro文件当中,这样就可以了。

\subsection{Qtcreator子项目}
在一个较大的工程项目当中,会有多个子项目,同时编译得到最后的文件。这在vs当中被称为solution。qtcreator当中也有类似的功能。

主要的操作是:在顶部的pro文件当中设置模板TEMPLATE为subdirs模式;然后设置SUBDIRS,添加指定的工程模块;使用"CONFIG+=ordered"配置为顺序编译。需要注意的是每一个单独的子项目都需要能够单独编译成功。为了方便,可以将这些工程的目标文件的生成路径设为一致。

这个子项目跟上面的pri文件不同,该文件对应的是一个项目很多时,包含很多模块进行简化。

\subsection{default构造函数}
在C++当中,类里面的数据大部分都是私有的,外部无法访问的,所以需要定义一些函数来进行赋值或者初始化的操作,这些函数可以称为接口。C++的类通过构造函数来对数据进行初始化操作,一个类当中可以有多个构造函数,这些函数的名称跟类名一致。当在类中没有定义任何构造函数时,C++会给你提供一个无参数的默认构造函数。但是,当你定义了一个含有参数的构造函数时,编译器就不再提供这个构造函数,如果你需要使用这个函数,就需要自己自已一个,否则,编译器就会报错。为了减轻程序员的负担,在C++的类声明时,使用"=default"来表示让编译器生成一个默认的构造函数,这样就不需要自己再去实现了。这个构造函数是在类的内部实现的,所以是一个内联函数。但是,vs2013之前的编译器都不支持这种语法,会报错。

需要注意的是,在vs2013当中遇到了这个错误:multiple versions of a defaulted special member functions are not allowed。原因可能是,使用default定义了一个默认构造函数,同时定义了一个含有参数的构造参数,但是这个参数的默认值被设为了null,编译器就认为出现了两个默认构造函数,所以会报错。

\subsection{error LNK2019: unresolved external symbol compress referenced in function}
如果使用Qtcreator出现了这个问题,恰好编译器是vs的话,有可能是一下原因:确定一下在pro文件当中,头文件列表和源文件列表中被调用函数的源文件是不是出现在了调用文件的后面,如果是的话,将它移动到前面去。

\subsection{构造函数}
1.父类没有声明构造函数

(1)子类也没有声明构造函数,则父类和子类均由编译器生成默认的构造函数;

(2)子类中声明了构造函数(无参或者带参),则子类的构造函数可以写成任何形式,不用顾忌父类的构造函数。在创建子类对象时,需要先调用父类的默认构造函数(编译器自动生成),然后再调用子类的构造函数。

2.父类只声明了无参数构造函数

如果子类的构造函数没有明显地调用父类的构造函数,则将会调用父类的无参构造函数。

3.父类只声明了带参数的构造函数

子类的构造函数必须显式得调用父类的构造函数。

4.父类同时声明了无参和有参构造函数

子类的构造函数调用其中一个即可。如果没有显式调用的话,会默认调用父类无参构造函数。

\subsection{Texstudio与sumatrapdf反向搜索关联}

sumatrapdf是一款免费小巧的PDF阅读器,双击PDF的某一个区域,可以打开关联的tex文件的相应位置。对于winedit,"D:/My Program Files/CTEX/WinEdt/WinEdt.exe" "[Open(|\%f|);SelPar(\%l,8)]"
对于TeXstudio"D:/My Program Files/TeXstudio/texstudio.exe"  "\%f" -line \%l

\subsection{VS不同版本编译报错}
出现“无法找到文件MSVCP120D.DLL”的问题,问题在于某些使用的dll文件是vs2013生成的,而现在使用的vs版本不是2013,解决方法就是将这些dll在新的vs下重新编译。

\subsection{Qt国际化}
在软件当中,需要实现界面的中文显示,这里采用的是Qt的国际化方案,也就是提供针对软件界面文字的中国版本的翻译文件,将国家设为中国就可以了。

简单介绍一下实现的步骤(详细的可以参考网上的教程):首先,在你需要翻译的地方,字符串需要写在tr函数当中,然后在pro文件当中加入ts文件,接着运行lupdate命令,生成相应的ts文件;然后使用语言家软件打开ts文件,或者你直接打开ts文件进行编辑,设置好相应的翻译;完成之后,运行lrelease命令生成压缩版的翻译文件qm;然后就是怎么使用的问题了,网上有好几种方案,一种就是放在exe文件的运行目录,另一种就是将qm文件放到工程的资源文件当中,这样的好处就是避免了文件的丢失,别人的篡改等等,可以编译进二进制文件当中。而且,如果加入到资源文件当中,每次自动生成qm文件就还在那个位置,就不需要你把它再拷贝到exe目录里。

简单说一下,在load qm文件时遇到的坑。我已经把qm文件加入到了qrc文件当中,但是无论如何都不能正确的载入,网上搜索了各种教程也不行。qm文件其实也拷贝到了文件夹下,也是不行。后来在res文件夹下面新建了一个translations文件夹,然后qm文件都放在了里面,然后接着load就好使了,我也不知道是为什么。